\section{Introduction}


Neural networks algorigthms have already become a state-of-the-art solution for a wide range of tasks
in many different fileds, such as computer vision, natural language processings, speech recognition and many others.
Most of these tasks have in common the possibility of representing processed data in Euclidean domain. However,
a large number of learning tasks have to deal with data expressed with graph. Graphs allow 
to represent information about relationship between different objects, but requires another approach 
to apply neural networks algorithms, as grah data is non-Euclidean. 

To solve this issue researches developed a sepecial kind of neural networks, which can operate graph data, called 
Graph Neural Network, or GNN \cite{GNN}. Since the first introduction of GNN, a lot of new upgraded, capable and more powerful 
models models have been developed and used for many practical applications in such areas as
physics simulations <TODO cite>,  antibacterial discovery<TODO cite>,  traffic prediction<TODO cite>, fake news detection<TODO>, 
recommendation systems <TODO> and many others. 

%For example. one of the most popular application is recommender system at Pinterest company, called PinSage, introduced in
% ... and used in production.  <TODO: edit and cite>

% TODO: maybe about twitter

% <examples from here https://distill.pub/2021/gnn-intro/>
% <May add examples and pictures from here https://gordicaleksa.medium.com/how-to-get-started-with-graph-machine-learning-afa53f6f963a>


GNN models can be used to solve three general types of prediction tasks on graphs: 
node level, edge level and graph level tasks. Node level tasks aims to predict some propeprty of node, for example
to predict node label. Edge level tasks include tasks related to edge properties prediction, such as labels of edges. 
Edge level tasks allow to discover connection types between the entites in graph. The remaining group of tasks is graph level tasks.
Such tasks are related to predicting of graph level properties, e.g. class of the given graph, which is known as
graph classification problem. This is analogous to image classification task in computer vision field or
sentiment analysis tasks in natural language processing.


Graph classification problem is very commonly used for biomedical applications.

% TODO: from Intro of Netprovec or from here  https://arxiv.org/pdf/2101.06569.pdf


The main goal of the current work is to compare several different approaches to solve graph classification problem 
on biomedical data. <TODO: say about relevance (why biomed?)>. This approaches will include the usage of 
several graph convolutional networks architectures : GCN \cite{GCN}, <TODO: list all with cites>. Besides, 
several ways for generation of node feature matrix will be applied. For node feature matrix generation will be used
encoded information about labels and degrees of node, as well as node distance distibution matrix(NDD), proposed in \cite{Netpro2vec}
and used to obtain graph embeddings.

The paper is organized as follows. \hyperref[sec:datasets]{Section2} contains detailed description of used biomedical dataset.
Short review of GNN algorigthms and description of used algorigthms are presented in \hyperref[sec:models]{Section 3}.
\hyperref[sec:experiments]{Section 4} contains implementation details and experiments results. Finally, \hyperref[sec:conclusion]{Section 6} reports conclusion. 
