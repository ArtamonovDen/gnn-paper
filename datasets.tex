\section{Datasets}
\label{sec:datasets}


Some common words from NetPro2vec.

Then list datasets

Table with stats

Description for each dataset

Datasets from TIDatasets \cite{TUDataset}

Link https://chrsmrrs.github.io/datasets/docs/datasets/


\begin{center}
    \begin{table}
        \begin{tabular}{|c|c|c|c|c|}
            \hline
            aa & ss & aa & aa\\
            \hline
            \hline
            STERE & 2012 & 1000 & $[251 \sim 1200] \times [222 \sim 900]$ \\
            \hline

            \hline
        \end{tabular}
    \caption{ss}
    \label{tab:datasets}
    \end{table}
\end{center}

\begin{center}
    \begin{table}[]
        \scriptsize
        \begin{tabular}{|l|l|l|l|l|l|l|l|}
        \hline
                             & MUTAG & PROTEINS & ENZYMES & NCI-1 & DD & Brain fMRI & Kidney RNA Seq \\ \hline
        \# Graphs            &       &          &         &       &    &            &                \\ \hline
        \# Classes           &       &          &         &       &    &            &                \\ \hline
        \# Samples per class &       &          &         &       &    &            &                \\ \hline
        \# Vertices          &       &          &         &       &    &            &                \\ \hline
        Average \# edge      &       &          &         &       &    &            &                \\ \hline
        Average edge density &       &          &         &       &    &            &                \\ \hline
        Has edge weights     &       &          &         &       &    &            &                \\ \hline
        Has edge attrubutes  &       &          &         &       &    &            &                \\ \hline
        Average degree       &       &          &         &       &    &            &                \\ \hline
                             &       &          &         &       &    &            &                \\ \hline
        \end{tabular}
        \end{table}
\end{center}

Datasets used for the current work can be devided into two groups. The first one includes datasets from TUDatasets \cite{TUDataset} -
a large collection of graph datasets, introduced in \cite{TUDataset} and very common and widely used for benchmarking of GNN models.
In current work the following TUDataset are used: MUTAG, PROTEINS, ENZYMES, NCI-1, DD.


Small molecules
A common class of graph datasets consists
of small molecules with class labels representing, e.g., toxicity
or biological activity determined in drug discovery projects.
Here, a graph represents a molecule, i.e., nodes take the places
of atoms and edges that of chemical bonds. Consequently, the
labels encode atom and bond types, possibly with additional
chemical attributes. The graph models differ, e.g., in whether
hydrogen atoms are represented explicitly by nodes, and bonds
in aromatic rings are annotated accordingly.
s
Our collection contains small datasets commonly used in the
early graph kernel literature such as MUTAG (Debnath et al.,
1991) and PTC (Helma et al., 2001), medium-sized datasets,
e.g., NCI1 and NCI109 (Wale et al., 2008; Shervashidze et al.,
2011), as well as several large datasets derived from the TOX21
challenge 2014 or PUBCHEM (Kim et al., 2018). This includes
the eleven datasets from anticancer screen tests with different
cancer cell lines used by Yan et al. (2008) to demonstrate
the efficacy of classifiers based on significant graph patterns.
These datasets, the largest of which contains more than 79k


Bioinformatics. The datasets DD, ENZYMES and PROTEINS
represent macromolecules. Borgwardt et al. (2005) introduced
a graph model for proteins, where nodes represent secondary
structure elements and are annotated by their type, i.e., helix,
sheet, or turn, as well as several physical and chemical information. An edge connects two nodes if they are neighbors along
the amino acid sequence or one of three nearest neighbors in
space. Using this approach, the dataset ENZYMES was derived
from the BRENDA database (Schomburg et al., 2004). Here, the
task is to assign enzymes to one of the 6 EC top-level classes,
which reflect the catalyzed chemical reaction. Similarly, the
dataset PROTEINS was derived from (Dobson  Doig, 2003),
and the task is to predict whether a protein is an enzyme. The
dataset DD used by Shervashidze et al. (2011) is based on
the same data, but contains graphs, where nodes represent
individual amino acids and edges their spatial proximity

% TODO: add cites for each dataset
% #TODO: list datasets and describe each of them

Another group consists of two real medical datasets used for benchmarking Netpro2vec model in \cite{Netpro2vec}: Brain fMRI and
Kidney RNASeq datasets.

Brain fMRI dataset includes 124 real graphs derived in \cite{BrainFmri} from the Center for Biomedical Research Excellence (COBRE) dataset, which contains
functional magnetic resonance imaging (fMRI) time-series data. Brain fMRI dataset includes 124 graphs: 54 graphs from Schizophrenia subjects (Sch) and 70 graphs from
healthy controls (Ctrl). Each graph contains 263 nodes corresponding to different brain regions. The edges have weights representing Fisher-transformed correlation between
the fMRI time-series of the nodes after ranking.

Kidney RNASeq dataset consists of metabolic networks constructed by mapping gene expression data on the biochemical reactions 
extracted from the kidney tissue metabolic model. Gene expression data is obtained from TCGA-KIRC and TCGA-KIRP projects
of the Genomic Data Commons (GDC, https://portal.gdc.cancer.gov) portal. Datasets was prepared by authors of \cite{Netpro2vec}
Kidney RNASeq dataset contains 299 networks devided into three classes. Two of them corresponds to the two most common types(according to National Cancer Institute(https://www.cancer.gov/pediatric-adult-rare-tumor/rare-tumors/rare-kidney-tumors/clear-cell-renal-cell-carcinoma and https://www.cancer.gov/pediatric-adult-rare-tumor/rare-tumors/rare-kidney-tumors/papillary-renal-cell-carcinoma))
of kidney cancer: Clear Cell Renal Cell Carcinoma (159 samples) and Papillary Renal Cell Carcinoma (90 samples). The third classs Solid Tissue Normal represents the absence of pathology 
and contains 50 samples.

Both datasets were downloaded from cbs-group repository\footnote{\url{https://github.com/cds-group/GraphDatasets}} \footnote{\url{https://github.com/cds-group/Netpro2vec}}, authors of Netpro2vec model \cite{Netpro2vec}.
