\section{GNN models overview}
\label{sec:models}

% TODO: In this section, we will provide a short literature review from the first attempts to
% apply neural nets to graph data to modern approaches, which are spreadly used TODO: check that
% Besides, this section contains the description of the models, used in experiments section/



% GNN

Early attempts to adapt neural network architectures to operate with graph data 
were proposed in works (Frasconi et al., 1998; Sperduti  Starita, 1997). These models 
used recursive neural networks (RNN) architecture. However these methods could be applied
only to directed acyclic graphs, which were a serious restriction. <TODO: check word>.
Models known as Graph Neural Networks (GNNs) were propesed in  Gori et al. [2005], Scarselli et al. [2004, 2009].
These models generalized recursive neural networks to deal with more general classes of graphs.
Presented in \cite{GNN} GNN architecture is commonly refered to be a simple base GNN.

% <TODO: detailed about vanila GNN model, deal with undirected graphs, limitations>


Over the last decade, there have been proposed a huge number of new architectures and framework, which 
outperform <TODO check word> vanila GNNs. 



% TODO: list some examples of different approaches?
In this work <TODO list approaches we will use>


% CONVOLUTIONS

% Convolution based model, which allows to deal with graph data, was GCN model proposed in \cite{GCN}.
A popular approach <TODO> is convolution based architecture, firstly proposed in work (Le-
Cun et al., 1995).
Convolution neural networks (CNN) have been applied to a wide range of computer
vision tasks, such as <TODO examples and cites> and have proved their effectiveness
to operate with grid-like structure data, such as images.
Any grid can be seen as a graph of grid like structure. So the idea to extend and generalize
convolutions to graph data is very natural and % TODO: вызывала большйой интерес.

% TODO: про свёртки немножко написать

... such convolution has predefined reception field. Which means, that convolution kernel is 
applied to the neighbourhood of predefined size. In case of grid-like structure data, such as
images, the structure of the image doesn't change: each pixel has same number of neighbors. 
However, such property is not guaranteed[TODO] for graph data. Each node of the graph may have 
different number of neighbors, so neighbourhood structure differs from node to node. So,
to apply convolutions to graph data, it should be prepared to operate with neighborhood 
with different structure. % TODO: edit

% TODO add img conv_vs_graph conv from https://arxiv.org/abs/1901.00596

The solution of this issue was proposed in Spectral Network model Bruna et al. [2014]. In this work
convolution operation is defined using eigendecomposition of the graph Laplacian.


For a undirected graph $G = (V,E)$ with $|V|=n$ , graph Laplacian matrix  $L \in R^{n \times n}$ is defined as 
\[L = D-A\]

where $A$ is adjacency matrix of the graph, and $D$ is degree matrix.

% TODO: можно формулок для A и D написать

Symmetic normalized Laplacian is defined as 
\[L = D^{-\frac{1}{2}}LD^{-\frac{1}{2}} = I - D^{-\frac{1}{2}}AD^{-\frac{1}{2}}\]

% TODO: добавить картинку laplacian.png from distil

The convolution operation  as for a feature node vector $x \in R^N$ with a
filter $g_\theta = diag(\theta)$ parameterized by $\theta \in R^N$ is defined as:

\[Conv(g_\theta,x) = Ug_\theta(\Lambda)U^{T}x\]

where $U$ is the matrix of eigenvectors of the normalized Laplacian, $\Lambda$ is a diagonal matrix of 
eigenvalues from the spectral decomposition of normalized Laplacian: $L = I - D^{-\frac{1}{2}}AD^{-\frac{1}{2}} = U\Lambda U^{T}$.

Such approach to the definition of convolution on graph data is known as localized spectral filter. And used convolutions is known as 
spectral convolutions.
However, this convolution operation is very computational expensive as multiplication with matrix $U$ has complexity $O(n^2)$.


Authors of [Hammond et al. [2011]] susuggested to use approximation to calculate $Conv(g_\theta,x)$ using Chebyshev polinomials 
$T_k(x)$ up to $K^{th}$, which is defined as 
\[
    T_k (x) = 2xT_{k-1}(x) - T_{k-2}(x)
\]
where $T_0(x)=1$ an $T_1(x)=x$

So convolution of a signal $x$ with filter $g_\theta$ is suggested to be:
\[
    Conv(g_\theta,x) \approx \sum_{k=0}^{K}{\theta}_k T_k(\hat{L})x
\]
where
$\hat{L} = \frac{2}{\lambda_{max}}L-I_N$ and $\lambda_max$ stands for the largest eigenvalue of $L$.
The complexity of evaluating convolution in proposed way is $O(|E|)$, i.e. is linear in the number of edges.

In \cite{ChebNet} there was proposed ChebNet - a convolutional neural network on grpahs based on K-localized convolution.

Authors \cite{GCN} developed the idea of usage spectral convolution and presented GCN model. The key idea is to build
convolutional neural network by stacking multiple simplified convolution layers with limited parameter $K=1$.
Single layer of such network can be expressed as: 
\[
    Conv(g_\theta,x) \approx {\theta}_0 T_0(\hat{L})x + {\theta}_1 T_1(\hat{L})x = {\theta}_0 x + {\theta}_1 (\frac{2}{\lambda_{max}}L-I_N)x
\]

Then authors of \cite{GCN} suggested to approximate $\lambda_{max} \approx 2$, so the expresion turns into the following one:

\[
    Conv(g_\theta,x) \approx {\theta}_0 x + {\theta}_1 (L-I_N)x = {\theta}_0 x - {\theta}_1 D^{-\frac{1}{2}}AD^{-\frac{1}{2}}x
\]

However, instead of two parameters $\theta_0$ and $\theta_1$ authors suggested to use the single one - $\theta = \theta_0 = -\theta_1$. Thus it allows to
address overfitting problem and to simplify computations per layer. So, the follwoing expression is

\[
    Conv(g_\theta,x) \approx {\theta}(I_N + D^{-\frac{1}{2}}AD^{-\frac{1}{2}})x
\]

Such expression of the convolution operation has a flow. The multiplier  $D^{-\frac{1}{2}}AD^{-\frac{1}{2}}$ has eigenvalues in 
range $[0,2]$, so that it leads to numerical instabilities when its repeated applying as layers in neural network. To alleviate this
problem, authors propose to replace numerical unstable multiplier $D^{-\frac{1}{2}}AD^{-\frac{1}{2}}$
with $\tilde{D}^{-\frac{1}{2}}\tilde{A}\tilde{D}^{-\frac{1}{2}}$, where $\tilde{A}=A+I_N$ and $\tilde{D}_{ii}=\sum_{j}\tilde{A}_{ij}$.

The final rule for linear convolutional layer for GCN can be generalized as following. Let signal $X \in R^{N\times C}$ is a matrix of $C$-dimensional
feature vectors for each of $N$ nodes, and $\Theta \in R^{C \times F}$ is a matrix of filter parameters. So convoled signal $X' \in R^{N \times F}$ is 
expessed as 

\begin{equation}
    X' = \tilde{D}^{-\frac{1}{2}}\tilde{A}\tilde{D}^{-\frac{1}{2}}X\Theta
    \label{eq:final_gcn}
\end{equation}

So, GCN model allows to calculate node representations as vectors (embeddings) using layer-wise convolutional neural network.

We can rewrite the equation \ref{eq:final_gcn} in term of neural network layer. So, obtained the graph convolutional layer 
is the following


\begin{equation}
    H^{(l+1)} = ReLU(\tilde{D}^{-\frac{1}{2}}\tilde{A}\tilde{D}^{-\frac{1}{2}}H^{(l)}W^{(l)})
    \label{eq:final_gcn_nn}
\end{equation}

where $H^{(l)} \in R^{n \times d}$ is a hidden state matrix on layer $l$, $d$ is a dimension
of the convolution and $W^{(l)} \in R^{d \ time d}$ is a matrix of trainable parmeters of the layer $l$.

% TOOD: about RelU ?


% GAT

Along with spectral convolution approach, there are other ways to prepare node embeddings. One of such approaches, is 
attention-based architecture model called Graph Attention Networks (GAT) proposed in \cite{GAT}.
The main idea of this model is to apply attention mechanism to graph data. Based on attention models have become 
the most common solution for many sequence-based tasks [TODO (Bahdanau et al., 2015; Gehring et al., 2016). from GAT].
The main feature of attention mechanism is that it allows to focus on the most relevant parts of the input 
to make a desicion.

Authors of \cite{GAT} proposes a graph attentional layer, which is a building block for the arbitrary graph atteniton network.
The input of the layer is a set of node feature $ h=\{ h_1,\dots , h_N \}$, where $h_i \in R^F$, $N$ - number of nodes, $F$ - the number
of node features. As result of layer processing is a new set of node features with other dimension: $h' = \{ h'_1 \dots h'_N \}$.
Each layer computes attentions coefficients $\alpha_{ij}$ by applying attention mechanism function $\alpha : R^{F'} \times R^{F'} \rightarrow R$.
to input node features multiplied to the shared weight matrix $W \in R^{F'} \times R^{F}$:

\begin{equation}
    e_{ij} = \alpha(Wh_i, Wh_j)
    \label{eq:att_coeff}
\end{equation}

This coefficient indicates the importance of of node j's features to node i. The next step is performing masked attention to 
be able to compute only coefficients for a node is some defined node's neighborhood $N$ and normalizing coefficients by neighborhood:


\begin{equation}
   \alpha_{ij} = softmax_j(e_{ij}) = \frac{exp(e_{ij})}{\sum_{k \in N_i}exp(e_{ik})}
\end{equation}

The last element is an attention function $\alpha$ itself. Authors of \cite{GAT} proposed to use
single-layer feedforward neural network parameterized by $a \in R^{2F'}$ and apply LeakyReLU nonlinearity.

As a result, the coefficients computed by the attention mechanism can be expressed as
\begin{equation}
    \alpha_{ij} = \frac{exp(LeackyReLU(a^{T}[Wh_i || Wh_j]))}{\sum_{k \in N_i}LeackyReLU(a^{T}[Wh_i || Wh_k])}
    \label{eq:final_att_coef}
 \end{equation}

where $||$ denotes concatenation operation. When coefficients are obtained - we can apply them to corresponding nodes:

\begin{equation}
    h'_i = \sigma \left( \sum_{j \in {N_i}} \alpha_{ij} W h_j \right)
    \label{eq:final_att}
 \end{equation}

To stabilize training procedure, authors extends self-attention mechanism
to multi-head attention, proposed in \cite{AttentionIsAllYouNeed}.
The idea of multi-head attention is to train several representations for each nodes at the same layer. And the final embedding
is just a concatenation of the all produced ones. For $K$ independant heads  there is a following final representation of the node:


\begin{equation}
    h'_i = ||_{k=1}^{K}  \sigma \left( \sum_{j \in {N_i}} \alpha^{k}_{ij} W^{k} h_j \right)
    \label{eq:multihead}
\end{equation}

On the prediction (final) layer, instead of concatenation, authors suggest to employ averaging over all attention heads:

\begin{equation}
    h'_i =  \sigma \left( \frac{1}{K} \sum_{k=1}^{K} \sum_{j \in {N_i}} \alpha^{k}_{ij} W^{k} h_j \right)
    \label{eq:multihead2}
\end{equation}

This last step is called aggregation and show on %TODO.
% TODO: add picture from GAT

So, GAT model allows to assign different importances to nodes of a same neighborhood which distinguishes it favorably from GCNs
model. Besides GAT model has another feature: the computation of the node-neighbor pairs is parallelizable thus the operation
of attention calculating is very efficient.

% GCN2

Both GCN\cite{GCN} and GAT\cite{GAT} models are shallow. Attempts to stacking more layers and to add non-linearity
tends to degrade the performance of these models \cite{GCNII} because of over-smoothing phenomena, described in \cite{OverSmoothing}.
Over-smoothing suggests that node representations becomes indistinguishable as thaey are inclined to converge 
to a certain value. This fact assigns restrictions on the number of layers in the network and as a result
limitations network's ability to extract information from high-order neighbors.


% TODO: later about resnet
% To overcome the isuue of network depth limit, in computer vision filed there has beed proposed ResNet model \cite{ResNet},
% based on residual connections between different layers of the network. 

Authors of \cite{GCNII} extends GCN model with two modifications to prevent over-smoothing issue. Proposed architecture 
called Graph Convolutional Network via Initial residual and Identity mapping (GCNII).

The $l$-th layer of GCNII is defined as the following:

\begin{equation}
    H^{l+1} = ReLU\left( ((1-\alpha_{l})\tilde{P}H^{l} + \alpha_{l}H^{0}) ((1-\beta_{l})I_n + \beta_{l} W^{l})  \right)
    \label{eq:gcnii_layer}
\end{equation}

where $\alpha_l$ and $\beta_l$ are hyperparameters.
and $\tilde{P} = \tilde{D}^{-\frac{1}{2}}A\tilde{D}^{-\frac{1}{2}}$ is the graph 
convolution matrix with the renormalization trick used in equation \ref{eq:final_gcn} inside the GCN's layers.

Comparing with GCN layer defined in \ref{eq:final_gcn_nn} GCNII has two modifications.
The first one is to add initial residual connection to the first layer $H^0$. It simulates 
residual connections between different layers of the network originally proposed in 
ResNet model \cite{ResNet} in the filed of computer vision. 
The initial residual connection in GCNII ensures that that the final representation
of each node retains at least a fraction of $\alpha$ from the input layer even if the network contains
many layers stacked. Authors of \cite{GCNII} suggest to set hyperparameter $\alpha_l = 0.1$ or $0.2$.


The second modification is to add identity mapping $I_n$ to each layer's matrix of weights.
Adding of identity mapping provides regularization on weight matrix $W^(l)$ to avoid over-fitting.
The value of $\beta$ depends of the number of the layer. Authors suggest to set $\beta_l = \log(\frac{\lambda}{l}+1) \approx \frac{\lambda}{l}$,
where $\lambda$ is hyperparameter. 


Authors of \cite{GCNII} propose a new deep model GCN-II, which show a simple way to impove [TODO]

Recent paper \cite{GCNII}

% TODO: about GCNII



% Pooling

Originally, GCN, GAT and GCNII models were developed for solving node classification problem.
These methods allows to get node embedding and learn neural network to match it to the specified label.

However, all these methods can be easily applied for solving graph classification problem using pooling operation \cite{distillGCN}.
Pooling operation allows to aggregate information from final node embeddings and the apply predictor for final graph classification.
In this work as aggregation function simple mean pooling function will be used:

\begin{equation}
    r_i = \frac{1}{N_i}\sum_{n=1}^{N_i}x_n
    \label{eq:mean_pool}
\end{equation}

where embedding vector $r_i$ of graph $i$ is calculated as a maen of node embeddings $x_n$.

As a final classifier, simple 2-layer perceptron will be used:
\begin{equation}
    c_i = W_2(Relu(W_1r_i))
    \label{eq:final_classifier}
\end{equation}



%TODO: In the current paper ...